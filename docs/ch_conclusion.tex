\chapter{Further Work}

Currently there are three
more obvious ways to improve or extend NeTS functionality:
adding multigraph compatibility; adding IO functionality
to the library; and adding more algorithms to NeTS.

The one which will probably require most work is adapting
the library to handle multigraphs.
Although NeTS does not have such capability,
some of its functions and error handling were created with the possibility
of addressing multigraphs in a future update in mind.
As an example, we can look at the way edges are handled.
They are stored in a Map data structure, meaning each edge has a unique ID.
This was done so that in the case multigraphs are addressed,
their property of multiple edges between two nodes can be
handled by the already existing edge system.

To handle multigraphs, more work would also have to be done
in regards to the already existing algorithms detailed in the
Algorithms chapter.
While we were creating the functions for triplets, for instance,
we realized that addressing this algorithm for multigraphs
could mean not just exceptions, but the creation of an entirely new function
that is exclusive to multigraphs.

Adding IO options to the library would be an important update to NeTS.
Although only mentioned in passing before, NeTS already has
the capability to import and export networks to CSV files that
contain an adjacency matrix.
Further IO options would include the creation and customization of
network images. These images could be similar to
the images taken from Net20 that were
used in this document. Because of this precedent,
adapting code and ideas from Net20's image processing
functions could help with NeTS' update.

Finally, we can extend NeTS' functionality by adding new algorithms.
We already have a list of algorithms to implement in a
future release of the ilbrary. Two of interest here are
optimized quadruplets (4-cycle) algorithm,
and breadth-first-search (BFS).
BFS is an extremely important algorithm for applied graph theory.
It is used for path-finding. When a GPS tries to find the shortest path
from one place to another, it is most likely using BFS or a variation of it.
BFS ended up not making into NeTS' first release, but it was present
in Net20. Thus, the code from Net20 could be adapted and improved to fit
into NeTS.
The quadruplets algorithm refers to a 4-cycle search algorithm. Much like
the triplets algorithm present here, it searches for vertex cycles. However,
instead of looking for cycles of 3 nodes, it finds cycles of length 4.
We are interested in 4-cycle specifically, because
it is possible, just like we did for triplets, to create
algorithms for cycle-finding that are optimized for quadruplets specifically.
We could also add a general cycle-finding algorithm, but
it wouldn't be as fast as what we could create by targeting
4-cycles with an algorithm.